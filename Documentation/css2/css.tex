%% LaTeX-Beamer template for KIT design
%% by Erik Burger, Christian Hammer

%% title picture by Klaus Krogmann
%%
%% version 2.1
%%
%% mostly compatible to KIT corporate design v2.0
%% http://intranet.kit.edu/gestaltungsrichtlinien.php
%%
%% Problems, bugs and comments to
%% burger@kit.edu

\documentclass[18pt]{beamer}


%% SLIDE FORMAT

% use 'beamerthemekit' for standard 4:3 ratio
% for widescreen slides (16:9), use 'beamerthemekitwide'
\usepackage{listings}
\usepackage{templates/beamerthemekit}
% \usepackage{templates/beamerthemekitwide}
\usepackage{color}
\definecolor{editorGray}{rgb}{0.95, 0.95, 0.95}
\definecolor{editorOcher}{rgb}{1, 0.5, 0} % #FF7F00 -> rgb(239, 169, 0)
\definecolor{editorGreen}{rgb}{0, 0.5, 0} % #007C00 -> rgb(0, 124, 0)
\usepackage{upquote}
\lstdefinelanguage{JavaScript}{
	morekeywords={typeof, new, true, false, catch, function, return, null, catch, switch, var, if, in, while, do, else, case, break},
	morecomment=[s]{/*}{*/},
	morecomment=[l]//,
	morestring=[b]",
	morestring=[b]'
}

\lstdefinelanguage{HTML5}{
	language=html,
	sensitive=true, 
	alsoletter={<>=-},
	otherkeywords={
		% HTML tags
		<html>, <head>, <title>, </title>, <meta, />, </head>, <body>,
		<canvas, \/canvas>, <script>, </script>, </body>, </html>, <!, html>, <style>, </style>, ><
	},  
	ndkeywords={
		% General
		=,
		% HTML attributes
		charset=, id=, width=, height=,
		% CSS properties
		border:, transform:, -moz-transform:, transition-duration:, transition-property:, transition-timing-function:
	},  
	morecomment=[s]{<!--}{-->},
	tag=[s]
}

\lstset{%
	% Basic design
	backgroundcolor=\color{editorGray},
	basicstyle={\small\ttfamily},   
	frame=l,
	% Line numbers
	xleftmargin={0.75cm},
	numbers=left,
	stepnumber=1,
	firstnumber=1,
	numberfirstline=true,
	% Code design   
	keywordstyle=\color{blue}\bfseries,
	commentstyle=\color{darkgray}\ttfamily,
	ndkeywordstyle=\color{editorGreen}\bfseries,
	stringstyle=\color{editorOcher},
	% Code
	language=HTML5,
	alsolanguage=JavaScript,
	alsodigit={.:;},
	tabsize=2,
	showtabs=false,
	showspaces=false,
	showstringspaces=false,
	extendedchars=true,
	breaklines=true,        
	% Support for German umlauts
	literate=%
	{Ö}{{\"O}}1
	{Ä}{{\"A}}1
	{Ü}{{\"U}}1
	{ß}{{\ss}}1
	{ü}{{\"u}}1
	{ä}{{\"a}}1
	{ö}{{\"o}}1
}
%% TITLE PICTURE

% if a custom picture is to be used on the title page, copy it into the 'logos'
% directory, in the line below, replace 'mypicture' with the 
% filename (without extension) and uncomment the following line
% (picture proportions: 63 : 20 for standard, 169 : 40 for wide
% *.eps format if you use latex+dvips+ps2pdf, 
% *.jpg/*.png/*.pdf if you use pdflatex)

\titleimage{pink-blau}

%% TITLE LOGO

% for a custom logo on the front page, copy your file into the 'logos'
% directory, insert the filename in the line below and uncomment it

%\titlelogo{mylogo}

% (*.eps format if you use latex+dvips+ps2pdf,
% *.jpg/*.png/*.pdf if you use pdflatex)

%% TikZ INTEGRATION

% use these packages for PCM symbols and UML classes
% \usepackage{templates/tikzkit}
% \usepackage{templates/tikzuml}

% the presentation starts here

\title[CSS Part 2]{CSS Part 2}
\subtitle{Extend beauty!}
\author{Lena, Kristin, Charlotte}

% Bibliography

\usepackage[citestyle=authoryear,bibstyle=numeric,hyperref,backend=biber]{biblatex}
\addbibresource{templates/example.bib}
\bibhang1em

\begin{document}

% change the following line to "ngerman" for German style date and logos
\selectlanguage{ngerman}

%title page
\begin{frame}
\titlepage
\end{frame}

\section {Wie kann man unsere Webseite noch sch"oner machen?!}
\begin{frame}{Wie kann man unsere Webseite noch sch"oner machen?!}
\begin {itemize}
\item Men"uleiste
\begin{itemize}
\item die immer \glqq oben \grqq an der Seite bleibt
\end{itemize}
\pause
\item mit Regenb"ogen
\begin{itemize}
\item weil Regenb"ogen alles sch"oner machen!
\end{itemize}
\end {itemize}
\end{frame}

\begin{frame}[fragile]{Ein div f"ur unsere Men"uleiste}
Um eine Men"uleiste zu stylen, m"ussen wir sie erstmal anlegen!
\begin {itemize}
\item div mit der Klasse ''menu'' als erstes Element in unserem body
\item in das div 
\begin{itemize}
\item eine "Uberschrift
\end{itemize}
\end{itemize}
\pause
\textbf{Los geht's!}
\end{frame}

\begin{frame}[fragile]{Ein div f"ur unsere Men"uleiste}
\begin{lstlisting}
<body>
  <div class="menu">
    <h1>Science Camp Informatik</h1>
  </div>
  <div class="inhalt">
  ...
\end{lstlisting}
\end{frame}

\begin{frame}[fragile]{Jetzt geht's ans Styling!}
\begin{itemize}
\item Mit der CSS Konfiguration ''position: fixed'' legen wir eine feste Position vom Men"u auf der Seite fest
\item Damit es ganz oben ist, geben wir ihm au"serdem die Werte 0 f"ur die Attribute top und left
\item Au"serdem soll die Men"uleiste "uber die ganze Breite gehen
\end{itemize}
\pause
Probiere es aus! Wie sieht Deine Seite jetzt aus?
\end{frame}

\begin{frame}[fragile]{Jetzt geht's ans Styling!}
\begin{itemize}
\item Um die Men"uleiste vom Rest der Seite abzuheben setzen wir jetzt noch
\begin{itemize}
\item das Attribut ''color'' um eine Farbe der Men"uleiste festzulegen
\item das Attribut ''opacity'' \\
opacity hat als Wert eine Zahl zwischen 0.0  und 1.0. Dieser Wert bestimmt, wie transparent die Farbe des Elementes ist. 
\end{itemize}
\end{itemize}
\pause
Probiere es aus! Wie sieht Deine Seite jetzt aus?
\end{frame}

\begin{frame}[fragile]{Aufgabe - Style Deine Men"uleiste}
\begin{itemize}
\item Probiere unterschiedliche Werte f"ur color und opacity aus
\item Style die "Uberschrift in eurer Men"uleiste, indem Du ihr eine eigene Klasse gibst
\end{itemize}
\end{frame}

\begin{frame}[fragile]{Regenb"ogen auf unserer Webseite}
\begin{itemize}
\item Um Farbverl"aufe zu generieren, benutzen wir linear-gradient als Wert f"ur background
\begin{lstlisting}
.body {
  background: linear-gradient([winkel], [farbe1], [farbe2], [farbe3], ...);
  ...
} 
\end{lstlisting}
\item winkel ist hier die Richtung des Farbverlauf-Effekts. Werte sind zum Beispiel ''45deg'' f"ur einen Farbverlauf von unten links nach oben rechts
\item Es k"onnen beliebig viele Farben gesetzt werden. Den Wert der Farbe kann man wieder mit rgb(66, 134, 244) angeben \footnote{Google: Color Picker}
\end{itemize}
\end{frame}

\begin{frame}[fragile]{Aufgabe - Regenb"ogen auf unserer Webseite}
\begin{itemize}
\item Setze linear-gradient als Hintergrund des body-Elements
\item Spiele mit verschiedenen Farben. Schaffst Du es, einen Regenbogen zu generieren?
\end{itemize}
\end{frame}

\begin{frame}[fragile]{Regenb"ogen!}
\begin{figure}
\includegraphics[width=0.8\textwidth]{Rainbow} 
\end{figure}
\end{frame}

\end{document}
