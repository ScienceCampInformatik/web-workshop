%% LaTeX-Beamer template for KIT design
%% by Erik Burger, Christian Hammer
%% title picture by Klaus Krogmann
%%
%% version 2.1
%%
%% mostly compatible to KIT corporate design v2.0
%% http://intranet.kit.edu/gestaltungsrichtlinien.php
%%
%% Problems, bugs and comments to
%% burger@kit.edu

\documentclass[18pt]{beamer}


%% SLIDE FORMAT

% use 'beamerthemekit' for standard 4:3 ratio
% for widescreen slides (16:9), use 'beamerthemekitwide'
\usepackage{listings}
\usepackage{templates/beamerthemekit}
% \usepackage{templates/beamerthemekitwide}
\usepackage{color}
\definecolor{editorGray}{rgb}{0.95, 0.95, 0.95}
\definecolor{editorOcher}{rgb}{1, 0.5, 0} % #FF7F00 -> rgb(239, 169, 0)
\definecolor{editorGreen}{rgb}{0, 0.5, 0} % #007C00 -> rgb(0, 124, 0)
\usepackage{upquote}
\lstdefinelanguage{JavaScript}{
	morekeywords={typeof, new, true, false, catch, function, return, null, catch, switch, var, if, in, while, do, else, case, break},
	morecomment=[s]{/*}{*/},
	morecomment=[l]//,
	morestring=[b]",
	morestring=[b]'
}

\lstdefinelanguage{HTML5}{
	language=html,
	sensitive=true, 
	alsoletter={<>=-},
	otherkeywords={
		% HTML tags
		<html>, <head>, <title>, </title>, <meta, />, </head>, <body>,
		<canvas, \/canvas>, <script>, </script>, </body>, </html>, <!, html>, <style>, </style>, ><
	},  
	ndkeywords={
		% General
		=,
		% HTML attributes
		charset=, id=, width=, height=,
		% CSS properties
		border:, transform:, -moz-transform:, transition-duration:, transition-property:, transition-timing-function:
	},  
	morecomment=[s]{<!--}{-->},
	tag=[s]
}

\lstset{%
	% Basic design
	backgroundcolor=\color{editorGray},
	basicstyle={\small\ttfamily},   
	frame=l,
	% Line numbers
	xleftmargin={0.75cm},
	numbers=left,
	stepnumber=1,
	firstnumber=1,
	numberfirstline=true,
	% Code design   
	keywordstyle=\color{blue}\bfseries,
	commentstyle=\color{darkgray}\ttfamily,
	ndkeywordstyle=\color{editorGreen}\bfseries,
	stringstyle=\color{editorOcher},
	% Code
	language=HTML5,
	alsolanguage=JavaScript,
	alsodigit={.:;},
	tabsize=2,
	showtabs=false,
	showspaces=false,
	showstringspaces=false,
	extendedchars=true,
	breaklines=true,        
	% Support for German umlauts
	literate=%
	{Ö}{{\"O}}1
	{Ä}{{\"A}}1
	{Ü}{{\"U}}1
	{ß}{{\ss}}1
	{ü}{{\"u}}1
	{ä}{{\"a}}1
	{ö}{{\"o}}1
}
%% TITLE PICTURE

% if a custom picture is to be used on the title page, copy it into the 'logos'
% directory, in the line below, replace 'mypicture' with the 
% filename (without extension) and uncomment the following line
% (picture proportions: 63 : 20 for standard, 169 : 40 for wide
% *.eps format if you use latex+dvips+ps2pdf, 
% *.jpg/*.png/*.pdf if you use pdflatex)

\titleimage{learn-coding-online}

%% TITLE LOGO

% for a custom o on the front page, copy your file into the 'logos'
% directory, insert the filename in the line below and uncomment it

%\titlelogo{mylogo}

% (*.eps format if you use latex+dvips+ps2pdf,
% *.jpg/*.png/*.pdf if you use pdflatex)

%% TikZ INTEGRATION

% use these packages for PCM symbols and UML classes
% \usepackage{templates/tikzkit}
% \usepackage{templates/tikzuml}

% the presentation starts here

\title[Javascript Basics]{Javascript Basics}
%\subtitle{Something for XYZ 2009}
\author{Lena, Kristin, Charlotte}

% Bibliography

\usepackage[citestyle=authoryear,bibstyle=numeric,hyperref,backend=biber]{biblatex}
\addbibresource{templates/example.bib}
\bibhang1em

\begin{document}

% change the following line to "ngerman" for German style date and logos
\selectlanguage{ngerman}

%title page
\begin{frame}
\titlepage
\end{frame}

\section {Was ist Javascript}
\begin{frame}{Was ist Javascript}
\begin {itemize}
\item Programmiersprache für Websites
\item Einfach zu lernen 
\item Wird direkt beim Aufruf in Maschinensprache übersetzt und ausgeführt
\end {itemize}
\end{frame}

\section {Was ist Javascript}
\begin{frame}[fragile]{Javascript für die Website}
\begin {itemize}
\item Einbinden der Javascript-Datei in die HTML-Datei (index.html)
\item Muss ganz unten im Body definiert werden
\begin{lstlisting}
<script src="script.js"> </script>
\end{lstlisting}
\item \glqq script.js\grqq als Datei im selben Ordner anlegen
\end {itemize}
\end{frame}


\begin{frame}[fragile]{Buttons!}
\begin {itemize}
\item Button Element zur HTML Website hinzuf"ugen
\item onclick ruft Funktion nach Event \glqq auf Button geklickt\grqq auf
\begin{lstlisting}
<button onclick="zeigeName()">
Zeige Namen!</button>
\end{lstlisting}
\end {itemize}
\end{frame}

\begin{frame}[fragile]{Buttons! - Die was tun!}
\begin {itemize}
\item Funktion muss noch in Javascript implementiert werden
\item Allgemeiner Funktionsaufbau in Javascript:
\begin{lstlisting}
function funktionsName(){
	//hier kommen die Befehle hin
	// "//" am Anfang der Zeile markiert Kommentar
}
\end{lstlisting}
\end {itemize}
\end{frame}




\begin{frame}[fragile]{Buttons! - Die was tun!}
\begin {itemize}
\item Die Funktion aus dem Button-Element muss definiert werden (in script.js)
\item  Mit window.alert Popup angezeigen
\item In den Klammern in Anf"uhrungszeichen Text definieren
\begin{lstlisting}
function zeigeName(){
window.alert("Hallo!");
}
\end{lstlisting}
\end {itemize}
\end{frame}



\begin{frame}[fragile]{Aufgabe - euer erster Button}
\begin {itemize}
\item F"ugt eurer Website einen Button hinzu
\item Lasst eure Website euch Grü"sen, wenn ihr auf den Button klickt
\end {itemize}
\end{frame}


\begin{frame}[fragile]{Variablen in Javascript}
\begin{itemize}
\item Im Code k"onnen Informationen gespeichert werden
\item Texte, Zahlen, Kommazahlen, Objekte
\item hei"st: Variablen! 
\item K"onnen in Funktionen oder einfach au"serhalb von Funktionen in der Datei stehen ($\Rightarrow$global!)
\begin{lstlisting}
var name = "Lotti";
var alter = 23;
var pi = "3,14159";
\end{lstlisting}
\end{itemize}
\end{frame}

\begin{frame}[fragile]{Variablen und Buttons}
\begin{itemize}
\item Text vom Button kann auch in einer Variable gespeichert werden
\begin{lstlisting}
var name = "Lotti";
function zeigeName(){
	window.alert("Hallo " + name);
}
\end{lstlisting}
\end{itemize}
\end{frame}


\begin{frame}[fragile]{Variablen und Buttons}
\begin{itemize}
\item Variable kann durch einen anderen Button ver"andert werden
\item Dazu: neuer Button (im HTML)!
\begin{lstlisting}
<button onclick="aendereName()">
Ändere den Namen</button>
\end{lstlisting}
\end{itemize}
\end{frame}


\begin{frame}[fragile]{Variablen und Buttons}
\begin{itemize}
	\item und die Funktion in Javascript 
	\begin{lstlisting}
	function aendereName(){
	name = prompt("Aendere den Namen!","Gib hier Namen an");
	}
	\end{lstlisting}
	\item prompt macht ein Popup in dem man einen Text eingeben kann
\end{itemize}
\end{frame}

\begin{frame}[fragile]{Inhalt der Website "andern}
\begin{itemize}
\item HTML Elementen können IDs haben
\item aber auch anders wiedergefunden werden
\begin{lstlisting}
	<p id="name"> </p>
\end{lstlisting}
\item in Javascript kann über document.getElementById auf das Element zugegriffen werden
\end{itemize}
\end{frame}

\begin{frame}[fragile]{Inhalt der Website ändern}
\begin{itemize}
	\item aus dem Javascript code k"onnen Eigenschaften wie Text, Styling ... ver"andert werden
	\begin{lstlisting}
	document.getElementById("name").textContent = name;
	\end{lstlisting}
\end{itemize}
\end{frame}

\begin{frame}[fragile]{Aufgabe}
\begin{itemize}
	\item F"ugt eurer Website ein Element hinzu, in der Name, der zuletzt eingegeben wurde angezeigt wird. 
	\item In Javascript: sobald der Name geändert wird soll dies in dem neuen Element angezeigt werden
\end{itemize}
\end{frame}

\begin{frame}[fragile]{Javascript Kleinigkeiten}
\begin{itemize}
	\item Listen (= Arrays in JS) sind Sammlungen an Objekten,Texten,Zahlen....
	\begin{lstlisting}
	var x = 10;
	x++; // <=> x = x+1 
	x--; // <=> x = x-1
	console.log("Hallo"); //Gibt Hallo auf der Konsole, Konsole sieht man bei f12 im Browser
	\end{lstlisting}
\end{itemize}
\end{frame}

\begin{frame}[fragile]{Listen und Schleifen}
\begin{itemize}
		\begin{lstlisting}
	var list = []; // []  leere Liste
	var list2 = [1,5,10]; // Liste aus 1,5,10
	\end{lstlisting}
	\item Elemente einer Liste hinzuf"ugen: 
		\begin{lstlisting}
	var list = []; // [] zeigt leere Liste
	list.push(15); //liste enthält nun das Element 15
	\end{lstlisting}	
\end{itemize}
\end{frame}


\begin{frame}[fragile]{Listen und Schleifen}
\begin{itemize}
	\item Zugriff auf Listen in Javascript über Index
	\begin{lstlisting}
	var list = [10,20,30];
	list[0]; // = 10, Index beginnt bei 0!
	list.length; // Länge der Liste, höchster Index +1 
	\end{lstlisting}
\end{itemize}
\end{frame}

\begin{frame}[fragile]{Einschub: Konsole}
\begin{itemize}
	\item Konsole für Entwicklerausgaben
	\item Fehler/Probleme aus Javascript werden dort angezeigt
	\item Kann Funktionen der Javascript Datei aufrufen
	\item Einfach F12 im Browser und zum Tab \glqq Console\grqq
	\item mit console.log("text") kann aus einer Funktion Text in der Konsole ausgegeben werden
\end{itemize}
\end{frame}


\begin{frame}[fragile]{Listen und Schleifen}
\begin{itemize}
\item Schleifen wenn etwas mehrmals getahn werden soll
\item While schleife = \glqq Mach etwas solange eine Bedingung erfüllt ist\grqq
\begin{lstlisting}
var x = 1;
while(x < 10){
	console.log("x");
	x++ // das gleiche wie x = x+1;
}
\end{lstlisting}
\item Was passiert? 
\end{itemize}
\end{frame}

\begin{frame}[fragile]{Listen und Schleifen}
\begin{itemize}
	\item Schleifen können auch alle Elemente einer Liste durchgehen
	\begin{lstlisting}
	var list = ["Hallo ", "Science ", "Camp"];
	for(var text in list){
		console.log(text);
	}
	\end{lstlisting}
	\item Was passiert? 
\end{itemize}
\end{frame}

\begin{frame}[fragile]{Aufgabe}
\begin{itemize}
	\item F"ugt einen Button hinzu, der ein Popup "offnet, mit dem man ein Lieblingsessen eingeben kann.
	\item speichert das Gericht in einer Liste.
\end{itemize}
\end{frame}

\begin{frame}[fragile]{Aufgabe - L"osung}
\begin{itemize}
 	\item in HTML:
 	\begin{lstlisting}
    <button onclick="essenHinzufuegen()"> Füge ein Essen hinzu</button>
 	\end{lstlisting}
 	\item in Javascript:
 	\begin{lstlisting}
 	var list = [];
 	function essenHinzufuegen(){
 	var essen = prompt("Gebe ein Essen ein, das du gerne isst.", "Pizza!");
 	list.push(essen);
 	//zeigeEssen();
 	}
 	\end{lstlisting}
 	\item n"achster Schritt: Liste der Lieblingsessen auf der Seite anzeigen. 
 	\item Jemand eine Idee wie?
\end{itemize}
\end{frame}

\begin{frame}[fragile]{Zeige Essensliste an!}
\begin{itemize}
	\item in HTML leeres Aufz"ahlungselement:
	\begin{lstlisting}
		  <ul id ="essen"></ul>
	\end{lstlisting}
\end{itemize}
\end{frame}
\begin{frame}[fragile]{Zeige Essensliste an!}
\begin{itemize}
	\item In Javascript: 
	\item l"osche erst alle Elemente di
	\begin{lstlisting}
	function zeigeEssen() {
	var ul = document.getElementById("essen");
	while( ul.firstChild ){
		ul.removeChild( ul.firstChild );
	}
	//...
	}
	\end{lstlisting}
	\item l"oscht das erste Kind der Aufz"ahlung bis keines mehr da ist
\end{itemize}
\end{frame}

\begin{frame}[fragile]{Zeige Essensliste an!}
\begin{itemize}
	\item Aufz"ahlungselemente erzeugen in Javascript: 
	\begin{lstlisting}
		function zeigeEssen() {
		li.appendChild(document.createTextNode("Text"));
	}
	\end{lstlisting}
	\item Erzeugt einen Punkt der Aufzählung und f"ullt den mit Text
\end{itemize}
\end{frame}


\begin{frame}[fragile]{Zeige Essensliste an!}
\begin{itemize}
	\item l"oschen von Aufz"ahlungselemten
	\begin{lstlisting}
	function zeigeEssen() {
	//...
		var ul = document.getElementById("essen");
		while( ul.firstChild ){
		ul.removeChild( ul.firstChild );
		}
	}
	\end{lstlisting}
	\item l"oscht das erste Kind der Aufz"ahlung bis keines mehr da ist
\end{itemize}

\end{frame}

\begin{frame}[fragile]{Aufgabe}
\begin{itemize}
	\item Implementiert die Funktion zeigeEssen
	\item ruft sie jedesmal auf, wenn ein essen der Liste hinzugefügt wird.
	\item Fügt einen Button hinzu, der die Liste leeren soll (Lösche Essen)
	\item Implementiert die Funktion dazu
\end{itemize}
\end{frame}

\begin{frame}[fragile]{L"osung}
		\begin{lstlisting}
			function zeigeEssen() {
			var ul = document.getElementById("essen");
			while( ul.firstChild ){
				ul.removeChild( ul.firstChild );
			}
			for(var essen in list ){
				var li = document.createElement("li");
				li.appendChild(document.createTextNode
				(list[essen]));
				ul.appendChild(li);
			}
			}
		\end{lstlisting}
\end{frame}

\begin{frame}[fragile]{L"osung}
		\begin{lstlisting}
			function essenHinzufuegen(){
			var essen = prompt("Gebe ein Essen ein, das du gerne isst.", "Pizza!");
			list.push(essen);
			zeigeEssen();
			}
		\end{lstlisting}
\end{frame}

\begin{frame}[fragile]{L"osung}
\begin{itemize}
	\item im HTML: 
	\begin{lstlisting}
	<button onclick="loescheEssen()">Löschen</button>
	\end{lstlisting}
	\item im Javascript:
	\begin{lstlisting}
	function loescheEssen(){
		list = [];
		zeigeEssen();
	}
	\end{lstlisting}
\end{itemize}
\end{frame}

\begin{frame}[fragile]{Dropdown Menus und CSS-Manipulation}
\begin{itemize}
	\item Dropdown-Menu um Hintergrund auszusuchen
	\item Man braucht: Dropdownmenu, ein paar Bilder/Hintergr"unde und eine Javascript-Funktion 
\end{itemize}
\end{frame}

\begin{frame}[fragile]{Dropdown Menus}
\begin{itemize}
	\item onchange: Funktion wird bei Neuauswahl aufgerufen
	\item id um den Wert draus zu lesen
	\item Option Kind-Element zeigt Optionen

\begin{lstlisting}
	<select onchange="aendereHintergrund()"  id="hintergrundWechsler">
		<option value="blumen">Blumen</option>
		<option value="normal">Normal</option>
	</select>
\end{lstlisting}
\end{itemize}
\end{frame}

\begin{frame}[fragile]{Hinterg"unde}
\begin{itemize}
	\item m"ussen als Bilddateien in Ordner bilder/hintergrund
	
\end{itemize}
\end{frame}

\begin{frame}[fragile]{Hinterg"unde}
\begin{itemize}
	\item Javascript:
	\begin{lstlisting}
function aendereHintergrund(){
	var wahl = document.getElementById("hintergrundWechsler").value
	//...
}
	\end{lstlisting}
	
\end{itemize}
\end{frame}

\begin{frame}[fragile]{Fallunterscheidung in Javascript}
\begin{itemize}
	\item \glqq Wenn Bedingung wahr ist mache A, sonst B\grqq
	\item === vergleicht auf gleichen Inhalt
	\begin{lstlisting}
	text = "Hallo";
	if(text === "Hallo"){
		console.log(text);
	}else{
		console.log("kein Hallo!");
	}
	\end{lstlisting}
\end{itemize}
\end{frame}

\begin{frame}[fragile]{Hintergrund "andern }
	\begin{lstlisting}
	if(wahl === "blumen"){
		document.querySelector("body").style.backgroundImage = "url(bilder/hintergrund/blumen.png)";
		document.querySelector("body").style.backgroundSize = "auto";
	}else{
		document.querySelector("body").style.backgroundImage = "url(bilder/hintergrund/pink-blau.png)";
		document.querySelector("body").style.backgroundSize = "cover";
	}
	\end{lstlisting}
\end{frame}


\begin{frame}[fragile]{Aufgabe: Hintergrund "andern }
\begin{itemize}
	\item F"ugt Website Hintergrundwechsler hinzu
	\item Implementiert Funktion
	\item Bonus: F"ugt mehr Hintergründe hinzu!
\end{itemize}
\end{frame}
\end{document}
