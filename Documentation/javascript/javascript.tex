%% LaTeX-Beamer template for KIT design
%% by Erik Burger, Christian Hammer
%% title picture by Klaus Krogmann
%%
%% version 2.1
%%
%% mostly compatible to KIT corporate design v2.0
%% http://intranet.kit.edu/gestaltungsrichtlinien.php
%%
%% Problems, bugs and comments to
%% burger@kit.edu

\documentclass[18pt]{beamer}


%% SLIDE FORMAT

% use 'beamerthemekit' for standard 4:3 ratio
% for widescreen slides (16:9), use 'beamerthemekitwide'
\usepackage{listings}
\usepackage{templates/beamerthemekit}
% \usepackage{templates/beamerthemekitwide}
\usepackage{color}
\definecolor{editorGray}{rgb}{0.95, 0.95, 0.95}
\definecolor{editorOcher}{rgb}{1, 0.5, 0} % #FF7F00 -> rgb(239, 169, 0)
\definecolor{editorGreen}{rgb}{0, 0.5, 0} % #007C00 -> rgb(0, 124, 0)
\usepackage{upquote}
\lstdefinelanguage{JavaScript}{
	morekeywords={typeof, new, true, false, catch, function, return, null, catch, switch, var, if, in, while, do, else, case, break},
	morecomment=[s]{/*}{*/},
	morecomment=[l]//,
	morestring=[b]",
	morestring=[b]'
}

\lstdefinelanguage{HTML5}{
	language=html,
	sensitive=true, 
	alsoletter={<>=-},
	otherkeywords={
		% HTML tags
		<html>, <head>, <title>, </title>, <meta, />, </head>, <body>,
		<canvas, \/canvas>, <script>, </script>, </body>, </html>, <!, html>, <style>, </style>, ><
	},  
	ndkeywords={
		% General
		=,
		% HTML attributes
		charset=, id=, width=, height=,
		% CSS properties
		border:, transform:, -moz-transform:, transition-duration:, transition-property:, transition-timing-function:
	},  
	morecomment=[s]{<!--}{-->},
	tag=[s]
}

\lstset{%
	% Basic design
	backgroundcolor=\color{editorGray},
	basicstyle={\small\ttfamily},   
	frame=l,
	% Line numbers
	xleftmargin={0.75cm},
	numbers=left,
	stepnumber=1,
	firstnumber=1,
	numberfirstline=true,
	% Code design   
	keywordstyle=\color{blue}\bfseries,
	commentstyle=\color{darkgray}\ttfamily,
	ndkeywordstyle=\color{editorGreen}\bfseries,
	stringstyle=\color{editorOcher},
	% Code
	language=HTML5,
	alsolanguage=JavaScript,
	alsodigit={.:;},
	tabsize=2,
	showtabs=false,
	showspaces=false,
	showstringspaces=false,
	extendedchars=true,
	breaklines=true,        
	% Support for German umlauts
	literate=%
	{Ö}{{\"O}}1
	{Ä}{{\"A}}1
	{Ü}{{\"U}}1
	{ß}{{\ss}}1
	{ü}{{\"u}}1
	{ä}{{\"a}}1
	{ö}{{\"o}}1
}
%% TITLE PICTURE

% if a custom picture is to be used on the title page, copy it into the 'logos'
% directory, in the line below, replace 'mypicture' with the 
% filename (without extension) and uncomment the following line
% (picture proportions: 63 : 20 for standard, 169 : 40 for wide
% *.eps format if you use latex+dvips+ps2pdf, 
% *.jpg/*.png/*.pdf if you use pdflatex)

\titleimage{logologo}

%% TITLE LOGO

% for a custom logo on the front page, copy your file into the 'logos'
% directory, insert the filename in the line below and uncomment it

%\titlelogo{mylogo}

% (*.eps format if you use latex+dvips+ps2pdf,
% *.jpg/*.png/*.pdf if you use pdflatex)

%% TikZ INTEGRATION

% use these packages for PCM symbols and UML classes
% \usepackage{templates/tikzkit}
% \usepackage{templates/tikzuml}

% the presentation starts here

\title[Javascript Basics]{Javascript Basics}
%\subtitle{Something for XYZ 2009}
\author{Lena, Kristin, Charlotte}

% Bibliography

\usepackage[citestyle=authoryear,bibstyle=numeric,hyperref,backend=biber]{biblatex}
\addbibresource{templates/example.bib}
\bibhang1em

\begin{document}

% change the following line to "ngerman" for German style date and logos
\selectlanguage{ngerman}

%title page
\begin{frame}
\titlepage
\end{frame}

\section {Was ist Javascript}
\begin{frame}{Was ist Javascript}
\begin {itemize}
\item Programmiersprache für Aktionen in Websiten
\item Einfach zu lernen 
\item Wird direkt beim Aufruf in Maschinensprache übersetzt und direkt ausgeführt
\end {itemize}
\end{frame}

\section {Was ist Javascript}
\begin{frame}[fragile]{Javascript für die Website}
\begin {itemize}
\item Zuerst:  einbinden der Datei die den Code enthält mit:
\begin{lstlisting}
<script src="script.js"> </script>
\end{lstlisting}
\grqq{script.js} als Datei anlegen
\end {itemize}
\end{frame}


\begin{frame}[fragile]{Buttons!}
\begin {itemize}
\item Button Element zur HTML Website hinzufügen. 
\item Name der Funktion die aus der Javascript-Datei ausgeführt werden soll, wenn man auf den Button klickt
\begin{lstlisting}
 <button onclick="zeigeName()">
 Zeige Namen!</button>
\end{lstlisting}
\item onClick ist ein Event, bei dessen auftreten die Funktion reagiert
\end {itemize}
\end{frame}



\begin{frame}{Der Mathetest}
\begin{block}{Aufgabe}
Lisa macht ein Austauschsemester in Australien. Um f"ur einen Mathetest zu lernen, l"ost sie Rechen-Aufgaben, die ihr eine Kommilitonin diktiert hat. Leider hat die Kommilitonin nicht gesagt, wie die Aufgaben geklammert sind. \\
Gegeben die Anzahl an Faktoren, wie viele verschiedene Wege gibt es diese zu klammern?
\end{block}
Beispiel:
\begin{itemize}
\item Gegeben: $\lbrace a, b, c, d \rbrace$
\item Gesucht: M"oglichkeiten f"ur Klammerung
\item $ a \left( b \left( c d \right) \right) $ , $\left( a b \right) \left( c d \right) $ , $\left( \left( a b \right) c \right) d$ , $\left( a \left( b  c \right) \right) d $ , $ a \left( \left( b  c \right) d \right) $
\end{itemize}
\end{frame}



\end{document}
