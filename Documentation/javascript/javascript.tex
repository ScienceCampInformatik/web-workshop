%% LaTeX-Beamer template for KIT design
%% by Erik Burger, Christian Hammer
%% title picture by Klaus Krogmann
%%
%% version 2.1
%%
%% mostly compatible to KIT corporate design v2.0
%% http://intranet.kit.edu/gestaltungsrichtlinien.php
%%
%% Problems, bugs and comments to
%% burger@kit.edu

\documentclass[18pt]{beamer}


%% SLIDE FORMAT

% use 'beamerthemekit' for standard 4:3 ratio
% for widescreen slides (16:9), use 'beamerthemekitwide'
\usepackage{listings}
\usepackage{templates/beamerthemekit}
% \usepackage{templates/beamerthemekitwide}
\usepackage{color}
\definecolor{editorGray}{rgb}{0.95, 0.95, 0.95}
\definecolor{editorOcher}{rgb}{1, 0.5, 0} % #FF7F00 -> rgb(239, 169, 0)
\definecolor{editorGreen}{rgb}{0, 0.5, 0} % #007C00 -> rgb(0, 124, 0)
\usepackage{upquote}
\lstdefinelanguage{JavaScript}{
	morekeywords={typeof, new, true, false, catch, function, return, null, catch, switch, var, if, in, while, do, else, case, break},
	morecomment=[s]{/*}{*/},
	morecomment=[l]//,
	morestring=[b]",
	morestring=[b]'
}

\lstdefinelanguage{HTML5}{
	language=html,
	sensitive=true, 
	alsoletter={<>=-},
	otherkeywords={
		% HTML tags
		<html>, <head>, <title>, </title>, <meta, />, </head>, <body>,
		<canvas, \/canvas>, <script>, </script>, </body>, </html>, <!, html>, <style>, </style>, ><
	},  
	ndkeywords={
		% General
		=,
		% HTML attributes
		charset=, id=, width=, height=,
		% CSS properties
		border:, transform:, -moz-transform:, transition-duration:, transition-property:, transition-timing-function:
	},  
	morecomment=[s]{<!--}{-->},
	tag=[s]
}

\lstset{%
	% Basic design
	backgroundcolor=\color{editorGray},
	basicstyle={\small\ttfamily},   
	frame=l,
	% Line numbers
	xleftmargin={0.75cm},
	numbers=left,
	stepnumber=1,
	firstnumber=1,
	numberfirstline=true,
	% Code design   
	keywordstyle=\color{blue}\bfseries,
	commentstyle=\color{darkgray}\ttfamily,
	ndkeywordstyle=\color{editorGreen}\bfseries,
	stringstyle=\color{editorOcher},
	% Code
	language=HTML5,
	alsolanguage=JavaScript,
	alsodigit={.:;},
	tabsize=2,
	showtabs=false,
	showspaces=false,
	showstringspaces=false,
	extendedchars=true,
	breaklines=true,        
	% Support for German umlauts
	literate=%
	{Ö}{{\"O}}1
	{Ä}{{\"A}}1
	{Ü}{{\"U}}1
	{ß}{{\ss}}1
	{ü}{{\"u}}1
	{ä}{{\"a}}1
	{ö}{{\"o}}1
}
%% TITLE PICTURE

% if a custom picture is to be used on the title page, copy it into the 'logos'
% directory, in the line below, replace 'mypicture' with the 
% filename (without extension) and uncomment the following line
% (picture proportions: 63 : 20 for standard, 169 : 40 for wide
% *.eps format if you use latex+dvips+ps2pdf, 
% *.jpg/*.png/*.pdf if you use pdflatex)

\titleimage{logologo}

%% TITLE LOGO

% for a custom logo on the front page, copy your file into the 'logos'
% directory, insert the filename in the line below and uncomment it

%\titlelogo{mylogo}

% (*.eps format if you use latex+dvips+ps2pdf,
% *.jpg/*.png/*.pdf if you use pdflatex)

%% TikZ INTEGRATION

% use these packages for PCM symbols and UML classes
% \usepackage{templates/tikzkit}
% \usepackage{templates/tikzuml}

% the presentation starts here

\title[Javascript Basics]{Javascript Basics}
%\subtitle{Something for XYZ 2009}
\author{Lena, Kristin, Charlotte}

% Bibliography

\usepackage[citestyle=authoryear,bibstyle=numeric,hyperref,backend=biber]{biblatex}
\addbibresource{templates/example.bib}
\bibhang1em

\begin{document}

% change the following line to "ngerman" for German style date and logos
\selectlanguage{ngerman}

%title page
\begin{frame}
\titlepage
\end{frame}

\section {Was ist Javascript}
\begin{frame}{Was ist Javascript}
\begin {itemize}
\item Programmiersprache für Aktionen in Websiten
\item Einfach zu lernen 
\item Wird direkt beim Aufruf in Maschinensprache übersetzt und direkt ausgeführt
\end {itemize}
\end{frame}

\section {Was ist Javascript}
\begin{frame}[fragile]{Javascript für die Website}
\begin {itemize}
\item Zuerst:  einbinden der Datei die den Code enthält mit:
\begin{lstlisting}
<script src="script.js"> </script>
\end{lstlisting}
\grqq{script.js} als Datei anlegen
\end {itemize}
\end{frame}


\begin{frame}[fragile]{Buttons!}
\begin {itemize}
\item Button Element zur HTML Website hinzufügen. 
\item Name der Funktion die aus der Javascript-Datei ausgeführt werden soll, wenn man auf den Button klickt
\begin{lstlisting}
 <button onclick="zeigeName()">
 Zeige Namen!</button>
\end{lstlisting}
\item onClick ist ein Event, bei dessen auftreten die Funktion reagiert
\end {itemize}
\end{frame}

\begin{frame}[fragile]{Buttons! - Die was tun}
\begin {itemize}
\item Funktion muss nun noch in Javascript implementiert werden
\item Allgemeines Funktionsaufbau:
\begin{lstlisting}
function funktionsName(){
	//hier kommen die Befehle hin
	// "//" am Anfang der Zeile markiert Kommentar
}
\end{lstlisting}
\end {itemize}
\end{frame}




\begin{frame}[fragile]{Buttons! - Die was tun!}
\begin {itemize}
\item Die Funktion aus dem Button-Element muss definiert werden
\item  mit window.alert Popup angezeigen
\item In den Klammern in Anführungszeichen Text der angezeigt werden soll definieren
\begin{lstlisting}
function zeigeName(){
window.alert("Hallo!");
}
\end{lstlisting}
\end {itemize}
\end{frame}



\begin{frame}[fragile]{Aufgabe - Buttons! - Die was tun!}
\begin {itemize}
\item Fügt eurer Website einen Button hinzu
\item Lasst eure Website euch Grüßen, wenn ihr auf den Button klickt
\end {itemize}
\end{frame}


\begin{frame}[fragile]{Variablen in Javascript}
\begin {itemize}
\item Im Code k"onnen Daten gespeichert werden
\item Texte, Zahlen, Kommazahlen, Objekte
\item hei"st: Variablen! 
\item K"onnen in Funktionen oder einfach au"serhalb stehen ($\Rightarrow$global!)
\begin{lstlisting}
var name = "Lotti";
var alter = 23;
var pi = "3,14159";
\end{lstlisting}
\end {itemize}
\end{frame}

\begin{frame}[fragile]{Variablen und Buttons}
\begin {itemize}
\item Der Text vom Button kann auch in einer Variable gespeichert werden
\begin{lstlisting}
var name = "Lotti";
function zeigeName(){
	window.alert("Hallo " + name);
}
\end{lstlisting}
\end {itemize}
\end{frame}


\begin{frame}[fragile]{Variablen und Buttons}
\begin {itemize}
\item Die Variable kann durch einen anderen Button verändert werden
\item Dazu: neuer Button (im HTML)!
\begin{lstlisting}
<button onclick="aendereName()">
Ändere den Namen</button>
\end{lstlisting}
\item und die Funktion in Javascript 
\begin{lstlisting}
function aendereName(){
name = prompt("Aendere den Namen!","Gib hier Namen an");
}
\end{lstlisting}
\item prompt macht ein Popup in dem man einen Text eingeben kann
\end {itemize}
\end{frame}
\end{document}
