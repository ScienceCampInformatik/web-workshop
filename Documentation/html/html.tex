%% LaTeX-Beamer template for KIT design
%% by Erik Burger, Christian Hammer
%% title picture by Klaus Krogmann
%%
%% version 2.1
%%
%% mostly compatible to KIT corporate design v2.0
%% http://intranet.kit.edu/gestaltungsrichtlinien.php
%%
%% Problems, bugs and comments to
%% burger@kit.edu

\documentclass[18pt]{beamer}


%% SLIDE FORMAT

% use 'beamerthemekit' for standard 4:3 ratio
% for widescreen slides (16:9), use 'beamerthemekitwide'
\usepackage{listings}
\usepackage{templates/beamerthemekit}
% \usepackage{templates/beamerthemekitwide}
\usepackage{color}
\definecolor{editorGray}{rgb}{0.95, 0.95, 0.95}
\definecolor{editorOcher}{rgb}{1, 0.5, 0} % #FF7F00 -> rgb(239, 169, 0)
\definecolor{editorGreen}{rgb}{0, 0.5, 0} % #007C00 -> rgb(0, 124, 0)
\usepackage{upquote}
\lstdefinelanguage{JavaScript}{
	morekeywords={typeof, new, true, false, catch, function, return, null, catch, switch, var, if, in, while, do, else, case, break},
	morecomment=[s]{/*}{*/},
	morecomment=[l]//,
	morestring=[b]",
	morestring=[b]'
}

\lstdefinelanguage{HTML5}{
	language=html,
	sensitive=true, 
	alsoletter={<>=-},
	otherkeywords={
		% HTML tags
		<html>, <head>, <title>, </title>, <meta, />, </head>, <body>,
		<canvas, \/canvas>, <script>, </script>, </body>, </html>, <!, html>, <style>, </style>, ><
	},  
	ndkeywords={
		% General
		=,
		% HTML attributes
		charset=, id=, width=, height=,
		% CSS properties
		border:, transform:, -moz-transform:, transition-duration:, transition-property:, transition-timing-function:
	},  
	morecomment=[s]{<!--}{-->},
	tag=[s]
}

\lstset{%
	% Basic design
	backgroundcolor=\color{editorGray},
	basicstyle={\small\ttfamily},   
	frame=l,
	% Line numbers
	xleftmargin={0.75cm},
	numbers=left,
	stepnumber=1,
	firstnumber=1,
	numberfirstline=true,
	% Code design   
	keywordstyle=\color{blue}\bfseries,
	commentstyle=\color{darkgray}\ttfamily,
	ndkeywordstyle=\color{editorGreen}\bfseries,
	stringstyle=\color{editorOcher},
	% Code
	language=HTML5,
	alsolanguage=JavaScript,
	alsodigit={.:;},
	tabsize=2,
	showtabs=false,
	showspaces=false,
	showstringspaces=false,
	extendedchars=true,
	breaklines=true,        
	% Support for German umlauts
	literate=%
	{Ö}{{\"O}}1
	{Ä}{{\"A}}1
	{Ü}{{\"U}}1
	{ß}{{\ss}}1
	{ü}{{\"u}}1
	{ä}{{\"a}}1
	{ö}{{\"o}}1
}
%% TITLE PICTURE

% if a custom picture is to be used on the title page, copy it into the 'logos'
% directory, in the line below, replace 'mypicture' with the 
% filename (without extension) and uncomment the following line
% (picture proportions: 63 : 20 for standard, 169 : 40 for wide
% *.eps format if you use latex+dvips+ps2pdf, 
% *.jpg/*.png/*.pdf if you use pdflatex)

\titleimage{logologo}

%% TITLE LOGO

% for a custom logo on the front page, copy your file into the 'logos'
% directory, insert the filename in the line below and uncomment it

%\titlelogo{mylogo}

% (*.eps format if you use latex+dvips+ps2pdf,
% *.jpg/*.png/*.pdf if you use pdflatex)

%% TikZ INTEGRATION

% use these packages for PCM symbols and UML classes
% \usepackage{templates/tikzkit}
% \usepackage{templates/tikzuml}
\usepackage{hyperref}
\usepackage[latin1]{inputenc}
\usepackage{xcolor}

% the presentation starts here

\title[HTML Basics]{HTML Basics}
%\subtitle{Something for XYZ 2009}
\author{Lena, Kristin, Charlotte}

% Bibliography

\usepackage[citestyle=authoryear,bibstyle=numeric,hyperref,backend=biber]{biblatex}
\addbibresource{templates/example.bib}
\bibhang1em

\begin{document}

% change the following line to "ngerman" for German style date and logos
\selectlanguage{ngerman}

%title page
\begin{frame}
\titlepage
\end{frame}

\section {Tagesziel}
\begin{frame}[fragile]{Tagesziel}
\begin {itemize}
\item Das soll am Tagesende entstanden sein: \url{file:///C:/Users/Kristin/Documents/ungeordneter%20Unikram/web-workshop-master(3)/web-workshop-master/Tag1Ergebnis/Workshop/index.html}
\end {itemize}
\end{frame}

\section {Wochenziel}
\begin{frame}[fragile]{Wochenziel}

%\begin{lstlisting}
%<script src="script.js"> </script>
%\end{lstlisting}

\end{frame}

\section{Allgemeiner Aufbau}
\begin{frame}[fragile]{Was ist HTML?}
Ein Beispiel:
	\begin{lstlisting}
	<!DOCTYPE html>
	<html>
	<head>
	<title>Page Title</title>
	</head>
	
	<body>
	<h1>My First Heading</h1>
	<p>My first paragraph.</p>
	</body>
	</html> 
	\end{lstlisting}
\end{frame}

\begin{frame}[fragile]{Was ist HTML?}
\begin {itemize}
\item ausgeschrieben: Hyper Text Markup Language (deutsch: Hypertext-Auszeichnungssprache)
\item beschreibt die Struktur von Webseiten in einer maschinenlesbaren Sprache
\item wichtig beim Arbeiten mit HTML sind Tags 
\end {itemize}
\end{frame}

\begin{frame}[fragile]{Allgemeiner Aufbau}
\fcolorbox{red}{white}{\parbox{\linewidth}{\begin{center}Wichtig: Solange wir HTML-Code schreiben, verwenden wir Tags.\end{center}}} \\
\begin{center}Jeder Tag ist wie folgt aufgebaut:\end{center}
\begin{lstlisting}
	<tagname>Inhalt des Tags...</tagname>
\end{lstlisting}
\fcolorbox{red}{white}{\parbox{\linewidth}{\begin{center}Wichtig: Jeder Tag, der ge\"offnet wird, muss wieder geschlossen werden au�er $<$!DOCTYPE html$>$!  (...)\end{center}}}
\end{frame}

%\begin{frame}[fragile]{Allgemeiner Aufbau}
%Das Ganze sieht dann beispielsweise so aus:
%\begin{lstlisting}
%	<!DOCTYPE html>
%	<head>
%	<title>Page Title</title>
%	</head>
%	
%	<body>
%	<h1>My First Heading</h1>
%	<p>My first paragraph.</p>
%	</body>
%	</html> 
%\end{lstlisting}
%\end{frame}

\begin{frame}[fragile]{Allgemeiner Aufbau}
\fcolorbox{red}{white}{\parbox{\linewidth}{\begin{center}Wichtig: Jeder HTML Code hat das gleiche Grundger\"ust mit grundlegenden Elementen!\end{center}}}
\begin {itemize}
\item Ihr beginnt immer mit dem Tag $<$!DOCTYPE html$>$ (...)
\item Danach $<$html$>$ (...)
\item Im $<$head$>$, dem Kopf der Datei, speichert ihr Metadaten (...)
\item Die Website braucht nat\"urlich auch einen Titel, den ihr mit $<$title$>$ vergeben k\"onnt.
\item Und zum Schluss kommt noch der K\"orper des Ganzen mit dem Tag $<$body$>$ (...)
\end {itemize}
\end{frame}

\begin{frame}[fragile]{Allgemeiner Aufbau}
Folgende zwei Tags werden im $<$body$>$ oft gebraucht:
\begin{itemize}
\item Das Element $<$h1$>$ definiert immer gro${\ss}$e \"Uberschriften und
\item das Element $<$p$>$ leitet immer Paragraphen ein.
\end{itemize}
\end{frame}

\begin{frame}[fragile]{Allgemeine Aufbau}
�bersichtliche Darstellung Grundger�st
\end{frame}

\begin{frame}[fragile]{Allgemeiner Aufbau}
	Aufgabe: Erstelle deine erste HTML Website (Erinnerung: Grundger�st). Binde dabei auch eine �berschrift und einen Paragraphen ein. \\
	(Inhalt schon vorgeben?) \\
	(Atom �ffnen, kurz erkl�ren!)
\end{frame}

\begin{frame}[fragile]{Allgemeiner Aufbau}
	L�sung
\end{frame}

\section{Kommentare}
\begin{frame}[fragile]{Kommentare}
nlkn
\end{frame}

\section{�berschriften}
\begin{frame}[fragile]{�berschriften}
\begin{itemize}
\item F\"ur \"Uberschriften verwendet ihr die Tags $<$h1$>$ bis $<$h6$>$:
\begin{lstlisting}
	<h1>�berschrift</h1>
\end{lstlisting}
\item Mit $<$h1$>$ markiert ihr die wichtigste �berschrift, mit $<$h6$>$ die am wenigsten Wichtigste.
%\item Dickere \"Uberschriften k�nnt ihr erstellen in dem ihr ein den Tag wie folgt erweitert:(das Style-Attribut font-size verwendet) zu Styles verschieben
%\begin{lstlisting}
%<h1 style="font-size:60px;">�berschrift</h1>
%\end{lstlisting} 
\item Um eure Website noch �bersichtlicher zu gestalten, k�nnt ihr waagerechte Linien verwenden mit dem Tag $<$hr$>$. So k�nnt ihr Text besser separieren.
\item Wie ihr �berschriften und Texte kreativer gestalten k�nnt, kommt sp�ter noch.
\end{itemize}
\end{frame}

\begin{frame}[fragile]{�berschriften}
Aufgabe: F�gt in eurer Website die �berschrift $"$Willkommen$"$ ein. (vorher schon?)
\end{frame}

\begin{frame}[fragile]{�berschriften}
L�sung
\end{frame}

\section{Paragraphen}
\begin{frame}[fragile]{Paragraphen}
\begin{itemize}
\item F�r Paragraphen verwendet ihr den Tag $<$p$>$.
\begin{lstlisting}
	<p>Paragraph</p>
\end{lstlisting}
\item Einen Zeilenumbruch k�nnt ihr mit $<$br$>$ herstellen.
\begin{lstlisting}
	<p>Zeilen-<br>umbruch</p>
\end{lstlisting}
\end{itemize}
\end{frame}

\begin{frame}[fragile]{Paragraphen}
Aufgabe: Binde deinen ersten Paragraphen in die Website ein, in dem ihr ganz kurz etwas zu eurer Website schreibt.
\end{frame}

\begin{frame}[fragile]{Paragraphen}
L�sung
\end{frame}

\section{Styles}
\begin{frame}[fragile]{Styles}
Bedeutung Attribut \\
Auch das Style-Attribut ist immer gleich aufgebaut:
\begin{lstlisting}
	<tagname style="eigenschaft:wert;">
\end{lstlisting}
Verweis CSS?
\end{frame}

\begin{frame}[fragile]{Styles}
Mit dem Style-Attribut k�nnt ihr jetzt unterschiedliche Elemente kreativer gestalten:
\begin{itemize}
\item die Hintergrundfarbe:
\begin{lstlisting}
<body style="backround-color:powderblue;">
	<h1>This is a heading</h1>
	<p>This is a paragraph</p>
</body>
\end{lstlisting}
\end{itemize}
\end{frame}

\begin{frame}[fragile]{Styles}
\begin{itemize}
\item die Textfarbe:
\begin{lstlisting}
<body>
	<h1 style="color:#006699;">This is a heading</h1>
	<p style="color:#006699;">This is a paragraph</p>
</body>	
\end{lstlisting}
\item die Schriftart:
\begin{lstlisting}
<body>
	<h1 style="font-family:verdana;">This is a heading</h1>
	<p> style="font-family:arial;">This is a paragraph</p>
</body>
\end{lstlisting}
\end{itemize}	
\end{frame}

\begin{frame}[fragile]{Styles}
\begin{itemize}
\item die Schriftgr��e:
\item Positionierung des Textes: 
\end{itemize}
\end{frame}

\begin{frame}[fragile]{Styles}
Aufgabe: Styles
\end{frame}

\begin{frame}[fragile]{Styles}
L�sung
\end{frame}

\section{Links}
\begin{frame}[fragile]{Links}
Inhalt...
\end{frame}

\begin{frame}[fragile]{Links}
Aufgabe: Links
\end{frame}

\begin{frame}[fragile]{Links}
L�sung
\end{frame}

\section{Bilder}
\begin{frame}[fragile]{Bilder}
Inhalt...
\end{frame}

\begin{frame}[fragile]{Bilder}
Aufgabe: Bilder
\end{frame}

\begin{frame}[fragile]{Bilder}
L�sung
\end{frame}

\section{Buttons}
\begin{frame}[fragile]{Buttons}
Inhalt...
\end{frame}

\begin{frame}[fragile]{Buttons}
Aufgabe: Buttons
\end{frame}

\begin{frame}[fragile]{Buttons}
L�sung
\end{frame}

\section{Listen}
\begin{frame}[fragile]{Listen}
Inhalt...
\end{frame}

\begin{frame}[fragile]{Listen}
Aufgabe: Listen
\end{frame}

\begin{frame}[fragile]{Listen}
L�sung
\end{frame}

\end{document}
